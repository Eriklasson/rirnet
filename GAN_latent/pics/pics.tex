% Title: Block diagram of Third order noise shaper in Compact Disc Players
% Author: Ramón Jaramillo
\documentclass[tikz,14pt,border=10pt]{standalone}
\usepackage{textcomp}
\usetikzlibrary{shapes,arrows}

\tikzset{pics/.cd,
myinput/.style args={#1#2#3#4#5#6#7}{
code={% #1=color, #2=x1, #3=y1, #4=edge style #5=x2,#6=y2, #7=label
\draw[color=#1]  (input)  +(#2,#3) edge[#4] +(#5,#6);
\draw[](input)  +(-9pt,-12pt) -- +(-9pt,12pt);
\draw[rotate=90]  (input)  +(-9pt,-12pt) --node[below=0.3cm,midway](){#7} +(-9pt,12pt);}}
 }

\begin{document}
\tikzstyle{colorful} = [fill=gray!10]
\tikzstyle{block} = [draw, colorful, rectangle, minimum height=3em, minimum width=3em]
\tikzstyle{ramp}=[block,pin={below:Ramp}]
\tikzstyle{sum} = [draw, colorful, circle, node distance=1cm]
%\tikzstyle{enc} = [draw, colorful, regular polygon, regular polygon sides=3, shape border rotate=30]
\tikzstyle{dec} = [draw, colorful, regular polygon, regular polygon sides=3, shape border rotate=90]
\tikzstyle{input} = [coordinate]
\tikzstyle{output} = [coordinate]
\tikzstyle{node} = [coordinate]
\tikzstyle{err} = [draw, colorful, rectangle, shape border rotate=90, minimum height=1.5cm]
\tikzstyle{enc} = [ trapezium,   draw,   
                    shape border rotate = 270, trapezium angle = 60,  
                    inner ysep=3pt, outer sep=1pt, inner xsep=1pt, 
                    text width = 2em, 
                    node distance=1cm, font=\large ]
\tikzstyle{dec} = [ trapezium,   draw,   
                    shape border rotate = 90, trapezium angle = 60,  
                    inner ysep=3pt, outer sep=1pt, inner xsep=1pt, 
                    text width = 2em, 
                    node distance=1cm, font=\large ]

\tikzset{every pin/.style={pin distance = 1mm}}
\tikzset{every pin edge/.style={draw=none}}


\newcommand{\suma}{\Large$+$}
\newcommand{\inte}{$\displaystyle \int$}
\newcommand{\derv}{\huge$\frac{d}{dt}$}

\begin{tikzpicture}[auto, thick, node distance=2cm, >=triangle 45]
\draw
	node at (0,0)[right=0mm, block, name=start]{\Large RIR}
	node [input, name=input1, right of=start] {} 
	node [enc, right of=input1, below=.25cm] (enc) {Enc}
	node [dec, right of=enc] (dec) {Dec}
	node [err, right of=dec, above=0cm] (err) {L1};
	

	\draw[-](start.east) to node{} (2.15,0);
	\draw[->]([xshift=1cm]start.east) to [out=270, in=180] node {} (enc);
 	\draw[-](enc) -- node {} (dec);
 	\draw[->, looseness=2](dec) to [out=0, in=180] ([yshift=-0.5cm]err.west);
 	\draw[->, looseness=0.75]([xshift=1cm]start.east) to [out=90,in=180] ([yshift=0.5cm]err.west) ;
 	\draw[->, looseness=1.5, dashed](err.south) to [out=270,in=270] (enc.south);
  	\draw[->, looseness=1.5, dashed](err.south) to [out=270,in=270] (dec.south);
	% Adder

\draw
	node at (0,-5)[right=0mm, block, name=start]{\Large RIR}
	node at (0,-7)[right=0mm, block, name=start_sig]{\Large SIG}
	node [input, name=input1, right of=start] {} 
	node [input, name=input2, right of=start_sig] {} 
	node [enc, right of=input1] (enc) {Enc}
	node [enc, right of=input2] (nn) {NN}
	node [err, right of=nn, above=.25cm] (err) {L2};
    
	\draw[->](start.east) to node{} (enc);
	\draw[->](start_sig.east) to node{} (nn);
	\draw[->, looseness=2](enc.east) to [out=0, in=180] node {} ([yshift=0.5cm]err.west);
	\draw[->, looseness=2](nn.east) to [out=0, in=180] node {} ([yshift=-0.5cm]err.west);
 	\draw[->, looseness=1.5, dashed](err.south) to [out=270,in=270] (nn.south);
 	
 \draw
	node at (0,-10.5)[right=0mm, block, name=start]{\Large SIG}
	node [input, name=input1, right of=start] {} 
	node [input, name=input2, right of=start_sig] {} 
	node [enc, right of=input1] (enc) {NN}
	node [dec, right of=enc] (dec) {Dec}
	node [block, right of=dec] (rir) {\Large RIR};
    
	\draw[->](start.east) to node{} (enc);
	\draw[->](start_sig.east) to node{} (nn);
 	\draw[-](enc) -- node {} (dec);
 	\draw[->](dec) -- node {} (rir);

\draw
	%node at (2.15,0) {\textbullet} 
	%node at (0,3){\textbullet}
	%node at (0,3){\textbullet}
    %node at (0,3){\textbullet}
    %node at (0,3){\textbullet}
    ;
    
	\node at (-0.5,1) [above=5mm, right=0mm] {\textsc{training the autoencoder}};
	\draw [color=gray,thick](-0.5,2) rectangle (7.5,-3);
	\node at (-0.5,-3) [below=5mm, right=0mm] {\textsc{training the regular neural network}};
	\draw [color=gray,thick](-0.5,-3) rectangle (7.5,-8.5);
	\node at (-0.5,-8.5) [below=5mm, right=0mm] {\textsc{using the framework}};
	\draw [color=gray,thick](-0.5,-8.5) rectangle (7.5,-11.5);
\end{tikzpicture}

\begin{tikzpicture}[auto, thick, node distance=2cm, >=triangle 45]
\draw
	node at (0,0)[right=0mm, block, name=start]{\Large RIR}
	node [input, name=input1, right of=start] {} 
	node [enc, right of=input1, below=.25cm] (enc) {Enc}
	node [dec, right of=enc] (dec) {Dec}
	node [err, right of=dec, above=0cm] (err) {L1};
	

	\draw[-](start.east) to node{} (2.15,0);
	\draw[->]([xshift=1cm]start.east) to [out=270, in=180] node {} (enc);
 	\draw[-](enc) -- node {} (dec);
 	\draw[->, looseness=2](dec) to [out=0, in=180] ([yshift=-0.5cm]err.west);
 	\draw[->, looseness=0.75]([xshift=1cm]start.east) to [out=90,in=180] ([yshift=0.5cm]err.west) ;
 	\draw[->, looseness=1.5, dashed](err.south) to [out=270,in=270] (enc.south);
  	\draw[->, looseness=1.5, dashed](err.south) to [out=270,in=270] (dec.south);
	% Adder

\draw
	node at (0,-5)[right=0mm, block, name=start]{\Large RIR}
	node at (0,-7)[right=0mm, block, name=start_sig]{\Large SIG}
	node [input, name=input1, right of=start] {} 
	node [input, name=input2, right of=start_sig] {} 
	node [enc, right of=input1] (enc) {Enc}
	node [enc, right of=input2] (nn) {NN}
	node [err, right of=nn, above=.25cm] (err) {L2}
	node [enc, right of=err, right=5mm] (disc) {Disc}
	node [err, right of=disc](err2){L3}
	;
    
	\draw[->](start.east) to node{} (enc);
	\draw[->](start_sig.east) to node{} (nn);
	\draw[->, looseness=1.5](enc.east) to [out=0, in=180] node {} ([yshift=0.5cm]err.west);
	\draw[-, looseness=1.5](enc.east) to [out=0, in=90] node {} ([yshift=0.5cm, xshift=0.3cm]err.east);
	\draw[-, looseness=1.5](nn.east) to [out=0, in=270] node {} ([yshift=-0.5cm, xshift=0.3cm]err.east);
	\draw[->, looseness=1.5](nn.east) to [out=0, in=180] node {} ([yshift=-0.5cm]err.west);
  	\draw[->, looseness=1.5, dashed](err.south) to [out=270,in=270] (nn.south);
  	\draw[->, looseness=1.5, dashed](err2.south) to [out=270,in=270] (disc.south);
 	\draw[-, looseness=1.9]([yshift=-0.5cm, xshift=0.25cm]err.east) to [out=55,in=180] (disc.west);
	\draw[->, looseness=1.5](disc.east) to [out=0, in=180] node {} ([yshift=-0.5cm]err2.west);
 	\draw[-, dotted, rounded corners=2mm]([xshift=-0.6cm]disc.west) -- ([yshift=1cm, xshift=-0.6cm]disc.west) -- ([yshift=1cm, xshift=0.3cm]disc.west);
 	\draw[->, dotted, looseness=2]([yshift=1cm, xshift=0.3cm]disc.west) to [out=0, in=180] ([yshift=0.5cm]err2.west);
 	\draw[->, looseness=1.3, dashed](disc.east) to [out=-20, in=270] node {} (nn.south);
 	
 	
 \draw
	node at (0,-10.5)[right=0mm, block, name=start]{\Large SIG}
	node [input, name=input1, right of=start] {} 
	node [input, name=input2, right of=start_sig] {} 
	node [enc, right of=input1] (enc) {NN}
	node [dec, right of=enc] (dec) {Dec}
	node [block, right of=dec] (rir) {\Large RIR};
    
	\draw[->](start.east) to node{} (enc);
	\draw[->](start_sig.east) to node{} (nn);
 	\draw[-](enc) -- node {} (dec);
 	\draw[->](dec) -- node {} (rir);

\draw
	%node at (2.15,0) {\textbullet} 
	%node at (0,3){\textbullet}
	%node at (0,3){\textbullet}
    %node at (0,3){\textbullet}
    %node at (0,3){\textbullet}
    ;
    
	\node at (-0.5,1) [above=5mm, right=0mm] {\textsc{training the autoencoder}};
	\draw [color=gray,thick](-0.5,2) rectangle (7.5,-3);
	\node at (-0.5,-3) [below=5mm, right=0mm] {\textsc{parallell training; neural network and discriminator}};
	\draw [color=gray,thick](-0.5,-3) rectangle (10,-8.5);
	\node at (-0.5,-8.5) [below=5mm, right=0mm] {\textsc{using the framework}};
	\draw [color=gray,thick](-0.5,-8.5) rectangle (7.5,-11.5);
\end{tikzpicture}

\end{document}